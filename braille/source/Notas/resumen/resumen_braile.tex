
\documentclass[draft]{article}

\usepackage[spanish]{babel}
\usepackage[utf8]{inputenc}
\usepackage[T1]{fontenc}
\usepackage{fullpage}
\usepackage[puttinydots]{braille}
%\usepackage[useemptybox]{braille}
%\usepackage[8dots]{braille}
%\usepackage[mirror]{braille}
%\usepackage[compact]{braille}

\usepackage[lmargin=2.54cm, tmargin=2.54cm, rmargin=2.54cm, bmargin=2.54cm]{geometry}

\newcommand{\mytable}[1]{%
    \enskip\begin{tabular}[t]{r|l} 
    \hline #1 \hline
    \end{tabular}\enskip}

% tabla de tildes

\newcommand{\mytildetable}[1]{%
	\enskip\begin{tabular}[t]{l|r|l} 
		\hline #1 \hline
	\end{tabular}\enskip}

% tabla de alfabeto griego

\newcommand{\myagtable}[1]{%
	\enskip\begin{tabular}[t]{l|r|l|l} 
		\hline #1 \hline
	\end{tabular}\enskip}

\newcommand{\mytablewith}[2]{%
    \enskip\begin{tabular}[t]{r|l} 
    \multicolumn{2}{c}{#1} \\
    \hline #2 \hline
    \end{tabular}\enskip}

\newcommand{\mytaglist}[1]{%
    \enskip\begin{tabular}[t]{@{}l@{}}
    #1
    \end{tabular}\enskip}



\title{Escritura con la Fuente Braille en \LaTeX{}}
\author{Ferreira Juan David}
\date{\today}



\begin{document}
	
\maketitle

\tableofcontents


\section{Braille Alphabets}

\begin{center}
	\mytable{
		\braille{a} & a 1 \\
		\braille{b} & b 2 \\
		\braille{c} & c 3 \\
		\braille{d} & d 4 \\
		\braille{e} & e 5 \\
		\braille{f} & f 6 \\
		\braille{g} & g 7 \\
		\braille{h} & h 8 \\
		\braille{i} & i 9 \\
		\braille{j} & j 0 \\
	}
    \mytable{
    	\braille{k} & k \\
    	\braille{l} & l \\
    	\braille{m} & m \\
    	\braille{n} & n \\
    	\braille{o} & o \\
    	\braille{p} & p \\
    	\braille{q} & q \\
    	\braille{r} & r \\
    	\braille{s} & s \\
    	\braille{t} & t \\
    }
	\mytable{
		\braille{u} & u \\
		\braille{v} & v \\
		\braille{w} & w \\
		\braille{x} & x \\
		\braille{y} & y \\
		\braille{z} & z \\
	}
\end{center}

\subsection{La letra ñ}


\begin{center}
	\mytildetable{
		ñ & \braille{{er}}     & \{er\} \\
	}
\end{center}

La letra \texttt{ñ} se escribe \verb|\braille{{er}}|

\braille{{er}}

La letra \texttt{Ñ} se escribe \verb|\braille{{Italic}{er}}|

\braille{{Italic}{er}}

ojo con escribir  \verb|\braille{er}| porque esto produce

\braille{er}

%\subsection{La letra ñ}
%
%
%\begin{center}
%	\mytildetable{
%		ñ & \braille{{enie}}     & \{enie\} \\
%	}
%\end{center}
%
%La letra \texttt{ñ} se escribe \verb|\braille{{enie}}|
%
%\braille{{enie}}
%
%La letra \texttt{Ñ} se escribe \verb|\braille{{Italic}{enie}}|
%
%\braille{{Italic}{enie}}
%
%ojo con escribir  \verb|\braille{enie}| porque esto produce
%
%\braille{er}



\subsection{La letra ü}

\begin{center}
	\mytildetable{
		ü & \braille{{ou}}     & \{ou\} \\
	}
\end{center}

La letra \texttt{ü} se escribe \verb|\braille{{ou}}|

\braille{{ou}} 

La letra \texttt{Ü} se escribe \verb|\braille{{Italic}{ou}}|

\braille{{Italic}{ou}}


\section{Prefijo indicador}

\begin{center}
	\mytildetable{
		Mayúscula&\braille{{Italic}}  & \{Italic\}   \\
	}
	\mytildetable{
		Números &\braille{{Number}} & \{Number\} \\	
		Letras  &\braille{{Letter}} & \{Letter\} \\
	}
\end{center}

\section{Signos de puntuación}

Los signos de puntuación comunes se escriben se escriben:



\section{Letra en español con diacrítico}

\begin{center}
	\mytildetable{
		á &\braille{{of}}     & \{of\}      \\
		é &\braille{{the}}    & \{the\}     \\
		í &\braille{{st}}     & / \{st\}    \\
		ó &\braille{{ing}}    & \{ing\}     \\
		ú &\braille{{with}}   & \{with\}    \\
		ü & \braille{{ou}}    & \{ou\}     \\
		ñ & \braille{{er}}    & \{er\}     \\
	}
\end{center}

%\section{Letra en español con diacrítico}

%\begin{center}
%	\mytildetable{
%		á& \braille{{ad}}   & \{ad\}      \\
%		é& \braille{{ed}}   & \{ed\}      \\
%		í& \braille{{id}}   & \{id\} /    \\
%		ó& \braille{{od}}   & \{od\}      \\
%		ú& \braille{{ud}}   & \{ud\}      \\
%		ü& \braille{{ou}}   & \{uu\}      \\
%		ñ& \braille{{er}}   & \{enie\}    \\
%	}
%\end{center}


\section{Signos de puntuación}

Los signos de puntuación comunes se escriben se escriben directamente cómo

\begin{center}
	\myagtable{
		$.$    &\braille{'}      & $(3)$  & '                  \\
		$,$    &\braille{,}      & $(2)$  & $,$                \\
		$;$    &\braille{;}      & $(23)$ & $;$                \\
		:      &\braille{:}      & $(25)$ & $:$                \\
		$-$    &\braille{-}      & $(36)$ & $-$                \\
		¿      &\braille{{en}}   & $(26)$ & \texttt{\{eta\}}   \\
		?      &\braille{{en}}   & $(26)$ & \texttt{\{eta\}}   \\	
	}
	\myagtable{
		¡      &\braille{!}      & $(235)$  & $!$\\
		$!$    &\braille{!}      & $(235)$  & $!$\\
		$"$    &\braille{{``}}   & $(236)$  & \texttt{\{``\}}  \\
		$``$   &\braille{{``}}   & $(236)$  & \texttt{\{``\}}  \\
		$($    &\braille{{gh}}   & $(126)$  & \texttt{\{gh\}}  \\
		$)$    &\braille{{ar}}   & $(345)$  & \texttt{\{ar\}}  \\
		$\dots$&\braille{'''}    & $(235)$  & \texttt{\{'''\}} \\
	}
\end{center}


\section{Operaciones aritméticas básicas}

\begin{center}
	\mytable{
		$+$&\braille{!} \\
		$-$&\braille{-} \\
		$\times$&\braille{{``}}    \\
		$.$&\braille{.}    \\
		$:$&\braille{.}    \\
	}
	\mytable{
		$=$&\braille{=} \\
		?&\braille{?} \\
		:&\braille{:}    \\
		.&\braille{.}    \\
		-&\braille{-} \\
	}
\end{center}


\section{Alfabeto Griego Braille}

\begin{center}
	\myagtable{
		\braille{{alpha}}    & $\alpha$   & $(4,1)$     & \texttt{\{alpha\}}   \\
		\braille{{beta}}     & $\beta$    & $(4,12)$    & \texttt{\{beta\}}    \\
		\braille{{gamma}}    & $\gamma$   & $(4,1245)$  & \texttt{\{gamma\}}   \\
		\braille{{delta}}    & $\delta$   & $(4,145)$   & \texttt{\{delta\}}   \\
		\braille{{epsilon}}  & $\epsilon$ & $(4,15)$    & \texttt{\{epsilon\}} \\
		\braille{{zeta}}     & $\zeta$    & $(4,1356)$  & \texttt{\{zeta\}}    \\
		\braille{{eta}}      & $\eta$     & $(4,156)$   & \texttt{\{eta\}}     \\   
		\braille{{theta}}    & $\theta$   & $(4,1456)$  & \texttt{\{theta\}}   \\
		\braille{{iota}}     & $\iota$    & $(4,24)$    & \texttt{\{iota\}}    \\ 
		\braille{{kappa}}    & $\kappa$   & $(4,13)$    & \texttt{\{kappa\}}   \\
		\braille{{lambda}}   & $\lambda$  & $(4,123)$   & \texttt{\{lambda\}}  \\
		\braille{{mu}}       & $\mu$      & $(4,134)$   & \texttt{\{mu\}}      \\	
	}
	\myagtable{
		\braille{{nu}}       & $\nu$      & $(4,1345)$  & \texttt{\{nu\}}      \\
		\braille{{xi}}       & $\xi$      & $(4,1346)$  & \texttt{\{xi\}}      \\
		\braille{{omicron}}  & $o$        & $(4,1345)$  & \texttt{\{omicron\}} \\
		\braille{{pi}}       & $\pi$      & $(4,1234)$  & \texttt{\{pi\}}      \\
		\braille{{rho}}      & $\rho$     & $(4,1235)$  & \texttt{\{rho\}}     \\
		\braille{{sigma}}    & $\sigma$   & $(4,234)$   & \texttt{\{sigma\}}   \\
		\braille{{tau}}      & $\tau$     & $(4,2345)$  & \texttt{\{tau\}}     \\
		\braille{{upsilon}}  & $\upsilon$ & $(4,136)$   & \texttt{\{upsilon\}} \\
		\braille{{phi}}      & $\phi$     & $(4,124)$   & \texttt{\{phi\}}     \\
		\braille{{chi}}      & $\chi$     & $(4,12346)$ & \texttt{\{chi\}}     \\
		\braille{{psi}}      & $\psi$     & $(4,13456)$ & \texttt{\{psi\}}     \\
		\braille{{omega}}    & $\omega$   & $(4,2456)$  & \texttt{\{omega\}}   \\
	}
\end{center}

\begin{center}
	\myagtable{
		\braille{{Alpha}}    & $A$        & $(45,1)$     & \texttt{\{Alpha\}}   \\
		\braille{{Beta}}     & $B$        & $(45,12)$    & \texttt{\{Beta\}}    \\
		\braille{{Gamma}}    & $\Gamma$   & $(45,1245)$  & \texttt{\{Gamma\}}   \\
		\braille{{Delta}}    & $\Delta$   & $(45,145)$   & \texttt{\{Delta\}}   \\
		\braille{{Epsilon}}  & $E$        & $(45,15)$    & \texttt{\{Epsilon\}} \\
		\braille{{Zeta}}     & $Z$        & $(45,1356)$  & \texttt{\{Zeta\}}    \\
		\braille{{Eta}}      & $H$        & $(45,156)$   & \texttt{\{Rta\}}     \\   
		\braille{{Theta}}    & $\Theta$   & $(45,1456)$  & \texttt{\{Theta\}}   \\
		\braille{{Iota}}     & $I$        & $(45,24)$    & \texttt{\{Iota\}}    \\ 
		\braille{{Kappa}}    & $K$        & $(45,13)$    & \texttt{\{Kappa\}}   \\
		\braille{{Lambda}}   & $\Lambda$  & $(45,123)$   & \texttt{\{Lambda\}}  \\
		\braille{{Mu}}       & $M$        & $(45,134)$   & \texttt{\{Mu\}}      \\	
	}
	\myagtable{
		\braille{{Nu}}       & $N$        & $(45,1345)$  & \texttt{\{Nu\}}      \\
		\braille{{Xi}}       & $\Xi$      & $(45,1346)$  & \texttt{\{Xi\}}      \\
		\braille{{Omicron}}  & $O$        & $(45,1345)$  & \texttt{\{Omicron\}} \\
		\braille{{Pi}}       & $\Pi$      & $(45,1234)$  & \texttt{\{Pi\}}      \\
		\braille{{Rho}}      & $P$        & $(45,1235)$  & \texttt{\{Rho\}}     \\
		\braille{{Sigma}}    & $\Sigma$   & $(45,234)$   & \texttt{\{Sigma\}}   \\
		\braille{{Tau}}      & $T$        & $(45,2345)$  & \texttt{\{Tau\}}     \\
		\braille{{Upsilon}}  & $\Upsilon$ & $(45,136)$   & \texttt{\{Upsilon\}} \\
		\braille{{Phi}}      & $\Phi$     & $(45,124)$   & \texttt{\{Phi\}}     \\
		\braille{{Chi}}      & $X$        & $(45,12346)$ & \texttt{\{Chi\}}     \\
		\braille{{Psi}}      & $\Psi$     & $(45,13456)$ & \texttt{\{Psi\}}     \\
		\braille{{Omega}}    & $\Omega$   & $(45,2456)$  & \texttt{\{Omega\}}   \\
	}
\end{center}



\section{Ejemplos}

\sloppy
\begin{itemize}
	\item Amistad 
	
	{\bf Grade 1:} Amistad \\
	\braille{{Italic}amistad}
	

	
\end{itemize}

\begin{itemize}
	\item jugar
	
	{\bf Grado 1:} jugar \\
	\braille{jugar}
	
	
	
\end{itemize}


\end{document}