
\documentclass[draft]{article}

\usepackage[spanish]{babel}
\usepackage[utf8]{inputenc}
\usepackage[T1]{fontenc}
\usepackage{fullpage}
\usepackage[puttinydots]{braille}
%\usepackage[useemptybox]{braille}
%\usepackage[8dots]{braille}
%\usepackage[mirror]{braille}
%\usepackage[compact]{braille}

\usepackage[lmargin=2.54cm, tmargin=2.54cm, rmargin=2.54cm, bmargin=2.54cm]{geometry}

\newcommand{\mytable}[1]{%
    \enskip\begin{tabular}[t]{r|l} 
    \hline #1 \hline
    \end{tabular}\enskip}

% tabla de tildes

\newcommand{\mytildetable}[1]{%
	\enskip\begin{tabular}[t]{l|r|l} 
		\hline #1 \hline
	\end{tabular}\enskip}

\newcommand{\mytablewith}[2]{%
    \enskip\begin{tabular}[t]{r|l} 
    \multicolumn{2}{c}{#1} \\
    \hline #2 \hline
    \end{tabular}\enskip}

\newcommand{\mytaglist}[1]{%
    \enskip\begin{tabular}[t]{@{}l@{}}
    #1
    \end{tabular}\enskip}



\title{Escritura con la Fuente Braille en \LaTeX{}}
\author{Ferreira Juan David}
\date{\today}



\begin{document}
	
\maketitle

\tableofcontents


\section{Braille Alphabets}

\begin{center}
	\mytable{
		\braille{a} & a 1 \\
		\braille{b} & b 2 \\
		\braille{c} & c 3 \\
		\braille{d} & d 4 \\
		\braille{e} & e 5 \\
		\braille{f} & f 6 \\
		\braille{g} & g 7 \\
		\braille{h} & h 8 \\
		\braille{i} & i 9 \\
		\braille{j} & j 0 \\
	}
    \mytable{
    	\braille{k} & k \\
    	\braille{l} & l \\
    	\braille{m} & m \\
    	\braille{n} & n \\
    	\braille{o} & o \\
    	\braille{p} & p \\
    	\braille{q} & q \\
    	\braille{r} & r \\
    	\braille{s} & s \\
    	\braille{t} & t \\
    }
	\mytable{
		\braille{u} & u \\
		\braille{v} & v \\
		\braille{w} & w \\
		\braille{x} & x \\
		\braille{y} & y \\
		\braille{z} & z \\
	}
\end{center}

\subsection{La letra ñ}


\begin{center}
	\mytildetable{
		ñ & \braille{{er}}     & \{er\} \\
	}
\end{center}

La letra \texttt{ñ} se escribe \verb|\braille{{er}}|

\braille{{er}}

La letra \texttt{Ñ} se escribe \verb|\braille{{Italic}{er}}|

\braille{{Italic}{er}}

ojo con escribir  \verb|\braille{er}| porque esto produce

\braille{er}

%\subsection{La letra ñ}
%
%
%\begin{center}
%	\mytildetable{
%		ñ & \braille{{enie}}     & \{enie\} \\
%	}
%\end{center}
%
%La letra \texttt{ñ} se escribe \verb|\braille{{enie}}|
%
%\braille{{enie}}
%
%La letra \texttt{Ñ} se escribe \verb|\braille{{Italic}{enie}}|
%
%\braille{{Italic}{enie}}
%
%ojo con escribir  \verb|\braille{enie}| porque esto produce
%
%\braille{er}



\subsection{La letra ü}

\begin{center}
	\mytildetable{
		ü & \braille{{ou}}     & \{ou\} \\
	}
\end{center}

La letra \texttt{ü} se escribe \verb|\braille{{ou}}|

\braille{{ou}} 

La letra \texttt{Ü} se escribe \verb|\braille{{Italic}{ou}}|

\braille{{Italic}{ou}}


\section{Prefijo indicador}

\begin{center}
	\mytildetable{
		Mayúscula&\braille{{Italic}}  & \{Italic\}   \\
	}
	\mytildetable{
		Números &\braille{{Number}} & \{Number\} \\	
		Letras  &\braille{{Letter}} & \{Letter\} \\
	}
\end{center}

\section{Signos de puntuación}

Los signos de puntuación comunes se escriben se escriben:



\section{Letra en español con diacrítico}

\begin{center}
	\mytildetable{
		á &\braille{{of}}     & \{of\}      \\
		é &\braille{{the}}    & \{the\}     \\
		í &\braille{{st}}     & / \{st\}    \\
		ó &\braille{{ing}}    & \{ing\}     \\
		ú &\braille{{with}}   & \{with\}    \\
		ü & \braille{{ou}}    & \{ou\}     \\
		ñ & \braille{{er}}    & \{er\}     \\
	}
\end{center}

%\section{Letra en español con diacrítico}

%\begin{center}
%	\mytildetable{
%		á& \braille{{ad}}   & \{ad\}      \\
%		é& \braille{{ed}}   & \{ed\}      \\
%		í& \braille{{id}}   & \{id\} /    \\
%		ó& \braille{{od}}   & \{od\}      \\
%		ú& \braille{{ud}}   & \{ud\}      \\
%		ü& \braille{{ou}}   & \{uu\}      \\
%		ñ& \braille{{er}}   & \{enie\}    \\
%	}
%\end{center}


\section{Signos de puntuación}

Los signos de puntuación comunes se escriben se escriben directamente cómo

\begin{center}
	\mytable{
		$($&\braille{{gh}} \\
		$)$&\braille{{ar}} \\
		$,$&\braille{,}    \\
		$;$&\braille{;}    \\
	}
	\mytable{
		¿&\braille{{``}} \\
		?&\braille{?} \\
		:&\braille{:}    \\
		.&\braille{'}    \\
		-&\braille{-} \\
	}
\end{center}


\section{Operaciones aritméticas básicas}

Signo de suma

Signo de resta

Multiplicación

División

Igualdad

Corte de una expresión

Desarrollo de las cuentas

Números negativos

Expresiones decimales periódicas

\begin{center}
	\mytable{
		$+$&\braille{!} \\
		$-$&\braille{-} \\
		$\times$&\braille{{``}}    \\
		$.$&\braille{'}    \\
		$:$&\braille{.}    \\
	}
	\mytable{
		$=$&\braille{=} \\
		?&\braille{?} \\
		:&\braille{:}    \\
		.&\braille{'}    \\
		-&\braille{-} \\
	}
\end{center}


\section{Ejemplos}

\sloppy
\begin{itemize}
	\item Amistad 
	
	{\bf Grade 1:} Amistad \\
	\braille{{Italic}amistad}
	

	
\end{itemize}

\begin{itemize}
	\item jugar
	
	{\bf Grado 1:} jugar \\
	\braille{jugar}
	
	
	
\end{itemize}


\end{document}